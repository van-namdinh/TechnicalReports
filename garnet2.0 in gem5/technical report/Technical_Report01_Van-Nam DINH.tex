\documentclass[13pt]{article}
\usepackage[english]{babel}
\usepackage{hyperref}
\usepackage{amsmath}
\usepackage{graphicx}
\graphicspath{{images/}}
\usepackage{parskip}
\author{Linus C. Brendel}								% Author
\date{\today}											% Date


\begin{document}
	
%%%%%%%%%%%%%%%%%%%%%%%%%%%%%%%%%%%%%%%%%%%%%%%%%%%%%%%%%%%%%%%%%%%%%%%%%%%%%%%%%%%%%%%%%

\begin{titlepage}
	\centering
	\vspace*{0.5 cm}
	\textsc{\LARGE Smart Integrated Systems Laboratory}\\[1.0 cm]	% University Name
	\textsc{\Large Working Progress for PhD course}\\[0.5 cm]				% Course Code
	\textsc{\large Technical Report}\\[0.5 cm]				% Course Name
	\textsc{\textbf{A Considering to the NoC's saturation points with Injection-rate base-on Gem5 simulator}}
	\rule{\linewidth}{0.2 mm} \\[0.4 cm]
	\rule{\linewidth}{0.2 mm} \\[1.5 cm]
	
	\begin{minipage}{0.4\textwidth}
		\begin{flushleft} \large
			\emph{Author: Van-Nam DINH}\\
		\end{flushleft}
	\end{minipage}~
	\begin{minipage}{0.4\textwidth}
		\begin{flushright} \large
			\emph{Student Number (ID):} 17028023									% Your Student Number
		\end{flushright}
	\end{minipage}\\[2 cm]
	
	
	\vfill
	
\end{titlepage}
%%%%%%%%%%%%%%%%%%%%%%%%%%%%%%%
	
	\tableofcontents
	\pagebreak
	
%%%%%%%%%%%%%%%%%%%%%%%%%%%%%%%

\section*{Abstract}
	%That include the targets and main contribution of working progess
	$>>>$
	
		
\section{Introduction}
	%It should include something that explain the context of the research.
	
\subsection{Gem5 Simulator}
	The Gem5 simulator is a modular platform that can be used for computer-systems architecture research. Gem5 includes many features such as Multiple interchangeable CPU models, A NoMali GPU model, Event-driven memory system, A trace-based CPU model that plays back elastic traces, which are dependency and timing annotated traces generated by a probe attached to the out-of-order CPU model; Homogeneous and heterogeneous multi-core; Multiple ISA support; \textbf{full-system capability}; Multi-system capability; Power and energy modeling and Co-simulation with SystemC.
	
	All of the above features can reference from \href{http://gem5.org/Main_Page}{link-ref-for-Gem5 features in detail}
	
	In addition, Gem5 is also known as a modular discrete event driven computer system platform, in which can be understand via the following points:
	\begin{enumerate}
		\item The components in Gem5 can be changed and rearranged, parameterized, extended or replaced to suit our design.
		\item Gem5 simulates the passing of time as a series of discrete events.
		
	\end{enumerate}
	
\subsection{Garnet2.0}
	

	
\section{Theory and definition | Related works}
%This part will present about the papers that related to my works

\section{Data and Results}
%This part should include the scenario

\section{Conclusion}
%don't know how to write this part

\section*{Acknowledgements and References}
%This part should write on one page.


%Reference	
	\newpage
	\bibliographystyle{plain}
	\bibliography{biblist}
	
\end{document}